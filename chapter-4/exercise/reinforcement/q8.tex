\documentclass{article}
\usepackage{amsmath}  % for mathematical symbols and environments
\usepackage{amsfonts} % for mathematical fonts
\usepackage{amssymb}  % for additional mathematical symbols
\usepackage[margin=1in]{geometry} % Set one inch margins
\usepackage{titlesec}
\usepackage{mdframed}
\usepackage{enumitem}

\renewcommand\thesection{Question \arabic{section}}
\renewcommand\thesection{Question \arabic{section}}


\titleformat{\subsection}[runin]{\large\bfseries}{}{0pt}{}[:]

\begin{document}

\title{Chapter 4: Recursion | Question 7}
\author{J.Z.W}
\date{November 29, 2023}

\maketitle

\setcounter{section}{6}

% question 7
\section{}

Isabel has an interesting way of summing up the values in a sequence \(A\)
of \(n\) integers, where \(n\) is a power of two. She creates a new sequence
\(B\) of half the size of \(A\) and sets \(B[i] = A[2i + 1]\), for 
\(i = 0, 1, ..., (n/2) - 1\). If \(B\) has size 1, then she outputs \(B[0]\).
Otherwise, she replaces A with B, and repeats the process. What is the running
time of her algorithm?

\subsection{Answer}

\begin{mdframed}

  Given the fact that the algorithms requires the use of the entire input size,
  the algorithm is at least \(O(n)\). Now for the entire input of \(A\) it is split
  into two, \(n/2\) and for the two parts of \(A\), they are added, thus \(n/2\) addition
  operations. Thus, the running time is \(O(n)\). 

\end{mdframed}





\end{document}
